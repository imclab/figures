\begin{tikzpicture}[thick,remember picture]
	
	\def\arbwidth{0.4cm}
	\def\arbheight{1.0cm}
	\def\arboffset{0.25cm}
	
	% An arbiter. Arguments:
	% 1: Name
	% 2: Node args
	% 3: Draw style
	\newcommand{\arb}[3]{
		\node (#1)
		      [minimum width=\arbwidth,minimum height=\arbheight, inner sep=0,#2]
		      {}
		      ;
		
		\coordinate (#1 in 1) at ($(#1.south west)!0.75!(#1.north west)$);
		\coordinate (#1 in 2) at ($(#1.south west)!0.25!(#1.north west)$);
		\coordinate (#1 out)  at (#1.east);
		
		\draw [#3]
		      (#1.south west)
		   -- (#1.north west)
		   -- ([yshift=-\arboffset]#1.north east)
		   -- ([yshift=\arboffset]#1.south east)
		   -- cycle
		      ;
	}
	
	% A buffer. Arguments:
	% 1: Name
	% 2: Node args
	% 3: Draw style
	\newcommand{\buf}[3]{
		\def\bufsize{0.2cm}
		\node (#1)
		      [draw,minimum width=\bufsize,minimum height=\bufsize, inner sep=0,#2,#3]
		      {}
		      ;
		
		\coordinate (#1 in)  at (#1.west);
		\coordinate (#1 out) at (#1.east);
	}
	
	% A buffer-arbiter pair. Arguments:
	% 1: Name
	% 2: Arbiter node args
	% 3: Draw style
	\newcommand{\arbbuf}[3]{
		\arb{#1 arb}{#2}{#3}
		\buf{#1 buf}{right=\arbbufgap of #1 arb}{#3}
		\draw [#3] (#1 arb out) -- (#1 buf in);
	}
	
	% Draw a line with two bends. Arguments:
	% 1: Start
	% 2: End
	% 3: Draw style
	% 4: Bend 1 style
	% 5: Bend 2 style
	\newcommand{\drawbent}[5]{
		\draw [#3] (#1) #4 ($(#1)!0.5!(#2)$) #5 (#2);
	}
	
	\def\arbbufgap{0.2cm}
	\def\arbvgap{0.2cm}
	\def\arbhgap{0.6cm}
	
	\def\routerhgap{0.5cm}
	
	% Output Buffers. Arguments:
	% 1: Output name
	% 2: Line first bend y offset
	% 3: Line first bend x offset
	\newcommand{\outbuf}[3]{
		\path let \p1=(#1 buf in), \p2=([xshift=0.5*\routerhgap]router.east) in (\x2,\y1) coordinate (#1 out);
		\buf{#1 out buf}{right=\routerhgap of #1 out}{}
		\draw ($(router.north east)!#2!(router.south east)$)
		   -- ++(#3,0)
		   |- (#1 out buf)
		      ;
	}
	
	\def\packetgenvgap{1.0cm}
	
	% The node itself
	\node (model) [draw, rectangle, inner sep=0.5cm] {
		\begin{tikzpicture}[thick,remember picture]
			% Level 2
			\arbbuf{e ne}{}{}
			\buf{e  buf}{left=\arbbufgap of e ne arb in 1}{} \draw (e  buf) -- (e ne arb in 1);
			\buf{ne buf}{left=\arbbufgap of e ne arb in 2}{} \draw (ne buf) -- (e ne arb in 2);
			
			\arbbuf{n w}{below=\arbvgap of  e ne arb}{}
			\buf{n buf}{left=\arbbufgap of n w arb in 1}{} \draw (n buf) -- (n w arb in 1);
			\buf{w buf}{left=\arbbufgap of n w arb in 2}{} \draw (w buf) -- (n w arb in 2);
			\arbbuf{sw s}{below=\arbvgap of n w arb}{}
			\buf{sw buf}{left=\arbbufgap of sw s arb in 1}{} \draw (sw buf) -- (sw s arb in 1);
			\buf{s  buf}{left=\arbbufgap of sw s arb in 2}{} \draw (s  buf) -- (sw s arb in 2);
			\arbbuf{proc}{below=\arbvgap of sw s arb}{draw=white}
			
			% Level 1
			\arbbuf{e ne n w}{right=\arbhgap of $(e ne buf)!0.5!(n w buf)$}{}
			\drawbent{e ne buf out}{e ne n w arb in 1}{}{-|}{|-}
			\drawbent{n w buf out}{e ne n w arb in 2}{}{-|}{|-}
			\arbbuf{sw s proc}{right=\arbhgap of $(sw s buf)!0.5!(proc buf)$}{}
			\drawbent{sw s buf out}{sw s proc arb in 1}{}{-|}{|-}
			\coordinate (proc in) at  (sw s proc arb in 2);
			
			% Level 0 (root)
			\arbbuf{root}{right=\arbhgap of $(e ne n w buf)!0.5!(sw s proc buf)$}{}
			\drawbent{e ne n w buf out}{root arb in 1}{}{-|}{|-}
			\drawbent{sw s proc buf out}{root arb in 2}{}{-|}{|-}
			
			% Router
			\node (router)
			      [draw, rectangle, right=\routerhgap of root buf out]
			      {Router}
			      ;
			\draw (root buf out) -- (router.west);
			
			\outbuf{e}{ 0.2}{0.3*\routerhgap + 0.1cm}
			\outbuf{ne}{0.3}{0.3*\routerhgap + 0.2cm}
			\outbuf{n}{ 0.4}{0.3*\routerhgap + 0.3cm}
			\outbuf{w}{ 0.5}{0.3*\routerhgap + 0.4cm}
			\outbuf{sw}{0.6}{0.3*\routerhgap + 0.3cm}
			\outbuf{s}{ 0.7}{0.3*\routerhgap + 0.2cm}
			
			% Packet Generator
			\node (packet gen)
			      [below=\packetgenvgap of sw s proc arb in 2, xshift=-0.5*\arbhgap]
			      [draw, rectangle, text width = 2cm, align=center, inner sep=0.2cm]
			      {Packet Generator}
			    ;
			
			\buf{packet gen buf}{above=\arbbufgap of packet gen.north}{}
			\draw (packet gen buf.north) |- (sw s proc arb in 2);
			\draw (packet gen.north) -- (packet gen buf.south);
			
			% Packet Consumer
			\path let \p1=(router.south)
			        , \p2=(packet gen.north)
			      in coordinate (packet con north) at (\x1,\y2);
			\node (packet con)
			      [below=0 of packet con north]
			      [draw, rectangle, text width = 2cm, align=center, inner sep=0.2cm]
			      {Packet Consumer}
			    ;
			\buf{packet con buf}{above=\arbbufgap of packet con.north}{}
			\draw (packet con.north) -- (packet con buf.south);
			\draw ($(router.north east)!0.8!(router.south east)$)
			   -- ++(0.3*\routerhgap + 0.1cm, 0)
			      coordinate (packet con router output)
			    ;
			\drawbent{packet con router output}{packet con buf.north}{}{|-}{-|}
			
		\end{tikzpicture}
	};
	
	
	\def\keylabelgap{0.3cm}
	
	% The Key
	\node [below=0.5cm of model, inner sep=0] {
		\begin{tikzpicture}[thick,remember picture]
			
			\arb{key arb}{}{}
			\node (key arb label) [right=\keylabelgap of key arb] {Arbiter};
			
			\buf{key buf}{right=of key arb label}{}
			\node (key buf label) [right=\keylabelgap of key buf] {Buffer};
			
			\node [left=of key arb] {Key:};
		\end{tikzpicture}
	};
	
	
	\def\pindist{1.0cm}
	
	% Input pins
	\draw (e  buf in) -- ++(-\pindist,0) node [left] {E};
	\draw (ne buf in) -- ++(-\pindist,0) node [left] {NE};
	\draw (n  buf in) -- ++(-\pindist,0) node [left] {N};
	\draw (w  buf in) -- ++(-\pindist,0) node [left] {W};
	\draw (sw buf in) -- ++(-\pindist,0) node [left] {SW};
	\draw (s  buf in) -- ++(-\pindist,0) node [left] {S};
	
	% Output pins
	\draw (e  out buf out) -- ++(\pindist,0) node [right] {E};
	\draw (ne out buf out) -- ++(\pindist,0) node [right] {NE};
	\draw (n  out buf out) -- ++(\pindist,0) node [right] {N};
	\draw (w  out buf out) -- ++(\pindist,0) node [right] {W};
	\draw (sw out buf out) -- ++(\pindist,0) node [right] {SW};
	\draw (s  out buf out) -- ++(\pindist,0) node [right] {S};
	
\end{tikzpicture}

